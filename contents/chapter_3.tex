\chapter{Chapter Title Goes Here 3}

\section{Section Title Goes Here 1}
As an example, we can cite the literature as follows~\cite{article-minimal,inbook-minimal}. Citations can also be made in this way~\cite{book-minimal,booklet-minimal,incollection-minimal}. Furthermore, equations in inline math mode are written like $F = ma$, while equations in display math mode are written as follows:
\begin{equation}
  \label{eq:c}
  F = ma.
\end{equation}
We can then refer to them as in (\ref{eq:c}). These colors can be customized in the preamble.

\lipsum[1-5]

\begin{figure}
  \centering
  \includegraphics[width=0.5\textwidth]{example-image-a}
  \caption{Sample figure 1.}
\end{figure}

\subsection{Subsection Title Goes Here 1}

\lipsum[6-9]

\begin{figure}
  \centering
  \includegraphics[width=0.5\textwidth]{example-image-b}
  \caption{Sample figure 2.}
\end{figure}

\paragraph{Paragraph Title Goes Here 1}

\lipsum[10]

\paragraph{Paragraph Title Goes Here 2}

\lipsum[11]

\subsection{Subsection Title Goes Here 2}

\lipsum[12-15]

\begin{table}
\centering
\caption{Sample Table 1.}
\begin{tabular}{lll}
  \hline
  Header 1 & Header 2 & Header 3 \\ \hline
  A & B & C \\
  D & E & F \\
  G & H & I \\ \hline
\end{tabular}
\end{table}

\section{Section Title Goes Here 2}

\lipsum[15-25]

\begin{table}
\centering
\caption{Sample Table 2.}
\begin{tabular}{lll}
  \hline
  Header 1 & Header 2 & Header 3 \\ \hline
  A & B & C \\
  D & E & F \\
  G & H & I \\ \hline
\end{tabular}
\end{table}